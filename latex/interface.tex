\section{Implementation}
\label{sec:implementation}

Bulk IO interface is a set of APIs that are built in the existing ROOT IO framework. The user can choose between regular APIs and Bulk IO APIs. We implement Bulk IO in three common use cases: \textit{TBranch}, \textit{TTreeReader} and \textit{RDataFrame}. We discuss about our interface design and integration in this section.

\subsection{Bulk IO in TBranch}

Listing \ref{bulkapi} shows Bulk IO API in TBranch in which two input arguments need to be parsed into the function: \textit{entry} and \textit{user\_buf}. the entry defines an event index number indicating which event the function is going to read. The \textit{user\_buf} parses an user-space TBuffer structure as a reference into the function. In the end of the function call, the \textit{user\_buf} should contain the whole basket of data that contains the inquiry event.

\lstset{
  basicstyle=\footnotesize\ttfamily, frame=single,
  xleftmargin=.05\textwidth, xrightmargin=.05\textwidth
}
\begin{lstlisting}[label={bulkapi}]
Int_t TBranch::GetBulkEntries(Long64_t entry, TBuffer &user_buf)
\end{lstlisting}
\vspace{-10pt}
\captionof{lstlisting}{Bulk API in TBranch}

%\vspace{5pt}

It is worthwhile to mention that \textit{GetBulkEntries} deserializes events on-the-fly when the data read into the \textit{user\_buf}. Thus no further manipulation is required for user applications. An user can later on access an event in the basket using the code snippet shown in Listing \ref{loop} where \textit{T} is the object type and \textit{idx} is the event index in the \textit{user\_buf}.

\lstset{
  basicstyle=\ttfamily, frame=single,
  xleftmargin=.08\textwidth, xrightmargin=.08\textwidth
}
\begin{lstlisting}[label={loop}]
 *reinterpret_cast<T*>(user_buf.GetCurrent())[idx]
\end{lstlisting}
\vspace{-10pt}
\captionof{lstlisting}{Loop over user\_buf}

\subsection{Bulk IO in TTreeReader}

\lstset{
  language=C++,
  keywordstyle=\color{blue},
  stringstyle=\color{red},
  commentstyle=\color{green},
%  caption=Access to Events using TTreeReader, 
  basicstyle=\ttfamily, frame=single,
  xleftmargin=.09\textwidth, xrightmargin=.09\textwidth
}
\begin{lstlisting}[label={ttreereader}]
TTreeReader myReader("T", hfile);
TTreeReaderValue<float> myF(myReader, "myFloat");
Long64_t idx = 0; 
Float_t sum = 1; 
while (myReader.Next()) { 
   sum += *myF;
}
\end{lstlisting}
\vspace{-10pt}
\captionof{lstlisting}{Access to Events using TTreeReader}

%\vspace{5pt}

\textit{TTreeReader} is an interface for an user to access simple object (primitives, arrays, etc.) in a ROOT file. \textit{TTreeReaderValue} is the interface to access primitives and \textit{TTreeReaderArray} is the interface to access arrays (each event is either an fixed-size or varialbe-size array). Listing \ref{ttreereader} shows a code sample that uses \textit{TTreeReaderValue} to read events (floats) from a file. \textit{TTreeReader} internally relies on \textit{GetEntry} to access events.

\lstset{
%  caption=Bulk API in TTreeReaderFast, 
  basicstyle=\footnotesize\ttfamily, frame=single,
  xleftmargin=0\textwidth,
  xrightmargin=0\textwidth
}
\begin{lstlisting}[label={serialized}]
Int_t TBranch::GetEntriesSerialized(Long64_t entry, TBuffer &user_buf)
\end{lstlisting}
\vspace{-10pt}
\captionof{lstlisting}{Bulk API in TTreeReaderFast}

%\vspace{5pt}

We design a Bulk API - \textit{GetEntriesSerialized} in TBranch shown in Listing \ref{serialized}. We introduce a new interface - TTreeReaderFast that uses \textit{GetEntriesSerialized} to function as TTreeReader. Unlike \textit{GetBulkEntries}, GetEntriesSerialized does not deserialize events while reading basket into \textit{user\_buf}. Instead, it waits until the user calls \textit{*myF}. Dereference operator invokes the appropriate deserialization code. 

\subsection{Bulk IO in RDataFrame}

During our work, Bulk IO is also integrated into RDataFrame \cite{rdataframe} which is a python Pandas \cite{pandas} like data analysis framework for ROOT users. RDataFrame provides a proxy interface - RDataSource (RDS). It allows RDF to read arbitrary data formats such as TTree, CSV, etc..

\lstset{
%  caption=Bulk API in RDataFrame, 
  basicstyle=\ttfamily,
  breaklines=true,
  xleftmargin=0.1\textwidth,
  xrightmargin=0.1\textwidth
}
\begin{lstlisting}[label={serializedcount}]
Int_t TBranch::GetEntriesSerialized(Long64_t entry, TBuffer &user_buf, TBuffer *count_buf)
\end{lstlisting}
\vspace{-10pt}
\captionof{lstlisting}{Bulk API in RDataFrame}

%\vspace{5pt}

We define a new Bulk API \textit{GetEntriesSerialized} shown in Listing \ref{serializedcount}. The only difference from Listing \ref{serialized} is that there is one more argument \textit{count\_buf}. Listing \ref{serialized} actually calls Listing \ref{serializedcount} and set the \textit{count\_buf} as nullptr. \textit{count\_buf} is used to store array length information when events in RDataFrame are arrays. Variable-size arrays need such information to deserialize the butter into multiple individual arrays.